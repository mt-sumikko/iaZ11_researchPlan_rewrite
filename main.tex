% 更新日2022年1月26日 by Tetsuaki Baba
\documentclass[a4paper]{jsarticle}
% 学部,修士は \documentclass[a4paper]{jsarticle}を利用して,\section で論文を構成してください.
% 博士は \documentclass[a4paper]{jsreport} を利用して, \chapter で論文を構成してください.

\usepackage{iapaper}
\usepackage[dvipdfmx]{graphicx}

\begin{document}
% 卒業論文の場合は \degreethesis
% 修士論文の場合は \masterthesis
% 博士論文の場合は \doctorthesis
\masterthesis %このサンプルでは卒業論文にしています


\title{研究計画(改訂版)}
\date{2022 年度}
\advisor{向井智彦} % 指導教員名を入れる
\IDnumber{22864614}
\Mauthor{隅山侑衣子} % 自分の名前を入れる.修士の場合は \Mauthor{氏名} に変更する.
\submissiondate{2022年7月1日}
\maketitle


\pagenumbering{roman} % 要旨はRoman書体で表示
\setcounter{page}{1} % 1から振り直す



\newpage

\pagenumbering{arabic}  % 論文本体はArabicで表示
\setcounter{page}{1} % ページ番号を1から振り直す

% 修士論文や博士論文等で記述ページが多い場合は,\section(節) ではなく,\chapter(章) を使ってください
\section{テーマ}
髪の毛等のパーツを意図的に動かすことで、感情表現を拡張するウェアラブルデバイスの提案

\section{そのテーマを希望する理由・動機}
感情表現の手段を増やすことによって、コミュニケーションを豊かにしたいからです。
人間の感情は主に顔面の表情や身振り手振りによって表現されますが、犬や猫のように動かせる耳や尻尾があれば、表現の幅が広がると考えます。また、筋肉の病気の有無にかかわらず、表情をうまく作れない人がいます。人間のコミュニケーションが主に顔を見て行われることを踏まえると、顔付近に感情表現の手段が増えることは、アドバンテージになると考えます。
つけ耳・つけ尻尾は既に製品化されており、中には動く機能を搭載したものもあります。しかし、耳や尻尾を模したものを「つける」と、それだけ目立つことになり、公の場での常用はしづらくなります。そこで、耳や尻尾の「形」をつけるのではなく、動かす対象は髪の毛などの既存のパーツとし、それを意図的に動かす「機能」を普段使いしやすいかたちで提案したいと考えます。

\section{そのテーマに関する研究の状況や背景}
人間につける尻尾の研究は「バランスをとる」「意図・感情を表現する」など複数の目的で展開されています。その成果物の多くは大きくて厳ついなど、座ることの多い生活では扱いにくい仕様になっています。そのような扱いづらさを概ね解消している作品に、脳波で動くしっぽ「shippo」があります。製品化されたシリーズ作品に、脳波で動く猫耳「necomimi」がありますが、仕様上、気軽に手に入る価格ではありません。いずれにせよ、これらの作品は動物らしさを表現したデザインになっているため、装着すると目立つという点と、脳波の計測が負担になるケースも考えられます。

\section{研究の目的} 
新しい表現方法およびコミュニケーションを提案することです。また、意図的に動かせる機能を獲得したパーツが、コミュニケーションに自然に生かせるかを検証することです。

\section{考えられる研究の方法} 
初めから地毛を扱うのは難しいため、エクステやかつらでの試作を経て、徐々に地毛にアプローチしていきます。 試作したツールを日常生活において実証実験し、操作方法や挙動を評価し、改善します。

\section{その研究の特色}
身体の物理的な拡張をベースとした機能強化ではなく、既存のパーツに対して機能拡張を目指している点です。また、関連研究の成果物が非日常向けの仕様が多いのに対して、日常的に利用可能なアウトプットを目標とし、新しいコミュニケーションの提案を目的としている点です。
\end{document}
