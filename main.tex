% 更新日2022年1月26日 by Tetsuaki Baba
\documentclass[a4paper]{jsarticle}
% 学部,修士は \documentclass[a4paper]{jsarticle}を利用して,\section で論文を構成してください.
% 博士は \documentclass[a4paper]{jsreport} を利用して, \chapter で論文を構成してください.

\usepackage{iapaper}
\usepackage[dvipdfmx]{graphicx}

\begin{document}
% 卒業論文の場合は \degreethesis
% 修士論文の場合は \masterthesis
% 博士論文の場合は \doctorthesis
\masterthesis %このサンプルでは卒業論文にしています


\title{研究計画(改訂版)}
\date{2022 年度}
\advisor{向井智彦} % 指導教員名を入れる
\IDnumber{22864614}
\Mauthor{隅山侑衣子} % 自分の名前を入れる.修士の場合は \Mauthor{氏名} に変更する.
\submissiondate{2022年7月1日}
\maketitle


\pagenumbering{roman} % 要旨はRoman書体で表示
\setcounter{page}{1} % 1から振り直す



\newpage

\pagenumbering{arabic}  % 論文本体はArabicで表示
\setcounter{page}{1} % ページ番号を1から振り直す

% 修士論文や博士論文等で記述ページが多い場合は,\section(節) ではなく,\chapter(章) を使ってください

%1:とりあえず敬体がキモい 記述の仕方を最低限改めて、あまりにも不本意な内容を治す 追記箇所は多少雑でもいい
%2:内容自体を差し障りない程度にアップデートし、文章としての質もさらに見直す
%3:研究計画書というフォーマットの性質上避けたのかもしれないけど、参照すべきものは記載しておく やっぱりいらないかもしれないが
\section{テーマ}
着用が目立たないウェアラブルデバイスによる髪の動的表現の開発と評価

\section{そのテーマを希望する理由・動機}
本研究テーマを希望する動機は、日常的な場面で利用可能な、感情や意図の表現手段を増やすことである。
人間の感情や意図は、発話や顔面の表情、身ぶり手ぶり等によって表現されるが、病気や習慣等の影響でこれらの手段をうまく使えない人もいる。そこで、犬や猫の尾のように感情や意図を表すことが一般に認知されている物を模倣すれば、表現の幅が広がると考える。ただし、模倣対象が人間の外観に調和しない場合は目立つため、使用シーンが限られる。そこで本研究では、目立たないことを条件とし、頭髪の毛束を尾に見立て、尾の表現を模倣する。
顔付近に表現手段が増えることは、人間のコミュニケーションが主に顔を見て行われることや、マスクの着用、および肩から上のみが映るビデオ通話を多用する現況を踏まえても、ハンディキャップの有無を問わず有用であると考える。
%動機のところが背景とか目的と被るのは、項目の性質上やむをえなさそうなので気にしないことにする


\section{そのテーマに関する研究の状況や背景}
%感情表現系デバイスへの言及
表現拡張系デバイスの例に、neurowearチームによる necomimi \cite{necomimi}やshippo \cite{shippo}がある。前者は猫の耳を、後者は尻尾をモチーフとしており、整体情報をセンシングして動く。その他の例を含め、多くの表現拡張系デバイスは,装着して目立つことを前提にデザインされてきた.それゆえ,目立たずに利用できない課題がある.感情等の表現対象が明確に現れることは,表現方法を拡張する目的にすらなり得る。しかし、表現媒体の様相として、その存在が目立つことに必然性はないと考える。そこで、耳や尻尾といった外観ごと身につけるのではなく、表現媒体は普段から人に備わっているものを活用し、それを動かす機能を与えるデバイスを提案する。表現媒体として、普段のコミュニケーションにおける視認性に優れた頭髪を採用する。
%しっぽ系の研究のことを一生懸命例示してるけど、ここは本質じゃないことをわかってない? 当時まだリサーチが甘かっただけで、本当はここでは髪系の先行事例のことを言わないといけないんだけども
頭髪を媒体とする関連研究では、その多くが髪に触れることによる情報の入力を扱っている\cite{smartzWig}\cite{Hairware}
人間につける尻尾の研究は
バランスをとる\cite{Augmenting Human With a Tail}、意図・感情を表現する\cite{Augmenting Human With a Tail}など、複数の目的で展開されている。その成果物の多くは大きくて厳ついなど、座ることの多い生活では扱いにくい仕様になっている。そのような扱いづらさを概ね解消している作品に、脳波で動くしっぽ「shippo」\cite{necomimi}があります。製品化されたシリーズ作品に、脳波で動く猫耳「necomimi」\cite{necomimi}がありますが、仕様上、気軽に手に入る価格ではありません。また、サーボモーター駆動のため、音がすること、長時間着用するには重量が重く、着用者の負担になる課題がある。なにより、これらの作品は動物らしさを表現したデザインになっている。装着すると目立つ点が、全事例に共通する課題である。%内容をガッツリいじってもしょうがないけど、差し支えない範囲でどうにかしたい

\section{研究の目的} 
目立たないという条件下で髪、とくに毛束を動かす手法の開発。編み込まない髪型、振るなどこれまで提案されて来なかった動きのバリエーションを展開する。髪を動かすことがどのような文脈、状況で受け入れられる 活用できるのかを評価する。

\section{考えられる研究の方法} 
髪を動かす技術的な検証段階については、ヘアエクステンションやかつらを用いた試作と、地毛で実現する有用性を吟味の上、地毛への展開を試みる。評価の段階では、プロトタイプを日常生活で用いることや、体験型のワークショップを通じて実証実験を行う。ユーザや観察者の意見・感想を収集して、操作方法や挙動を評価し、改善する。

\section{その研究の特色}
身体をハードウェア込みで拡張するのではなく、既存の身体部位に対して機能拡張を行うことをコンセプトとしている点が特色である。このアプローチにより、デバイス着用における心身への負荷を軽減できる。また、関連研究の成果物の多くが目立つ外観をしているのに対して、目立たないデバイスを目指す点、また髪を動かすことが日常的な文脈でいかに活かし得るのかを探求する点が特徴である。


\begin{thebibliography}{999}

\bibitem{necomimi}
\begin{flushleft}
  neurowear: necomimi, 2012・2021. 参照ページ:“necomimi”, neurowear,https://neurowear.com/necomimi/,(2022年3月20日最終閲覧)
\end{flushleft}

\bibitem{shippo}
  neurowear: shippo, 2012. 参照ページ:“脳波で動くしっぽ “shippo” ”,KILUCK:機楽株式会社,2012年10月15日,http://www.kiluck.co.jp/?p=61\&lang=ja(2022年3月20日最終閲覧)
  
 \bibitem{smartWig}
Tobita, H., and Kuzi, T. SmartWig: Wig-Based Wearable Computing Device for Communication and Entertainment. In Proceedings of the International Working Conference on Advanced Visual Interfaces (2012), AVI ’12, Association for Computing Machinery, pp. 299\UTF{2013}302. https://doi.org/10.1145/2254556.2254613
 \end{flushleft}

\bibitem{Augmenting Human With a Tail}
  Haoran Xie, Kento Mitsuhashi, Takuma Torii: Augmenting Human With a Tail, AH2019: Proceedings of the 10th Augmented Human International Conference 2019, Article No.: 35,pp.1\UTF{2013}7, 2019, https://doi.org/10.1145/3311823.3311847

\bibitem{Thanks tail}
八谷和彦:Thanks tail, 1996. 参照ページ:“ Thanks tail”,PetWORKs,2009年8月28日,http://www.petworks.co.jp/~hachiya/works/ThanksTail.html(2022年3月20日最終閲覧)

\bibitem{Thanks tail_movie}
YouTubeチャンネル「kazuhiko hachiya」:“ThanksTail (1996) サンクステイル - ”,YouTube,2009年8月28日,https://www.youtube.com/watch?v=47CX3hjpjRk(2022年4月13日最終閲覧)
  
\bibitem{Thanks tail article}
\begin{flushleft}
アイティメディア株式会社:“ウワサの車用“しっぽ”を動かしてきました”, ITmedia NEWS ,2004年12月25日,https://www.itmedia.co.jp/lifestyle/articles/0412/15/news033.html(2022年3月20日最終閲覧)
\end{flushleft}

\bibitem{kojien}
新村出編,広辞苑(第 6 版),岩波書店,2008

\bibitem{jpKnowledge}
\begin{flushleft}
株式会社ネットアドバンス:ジャパンナレッジ Lib,https://japanknowledge.com/library/(2022年3月17日最終閲覧)
\end{flushleft}

\bibitem{oyasan}
水瀬るるう:『大家さんは思春期!』,株式会社芳文社,第1巻 pp.57\UTF{2013}61,2013

\bibitem{servant_1}
高津カリノ:『サーバント×サービス』,株式会社スクウェア・エニックス,第2巻 p.92,2012

\bibitem{servant_2}
高津カリノ:『サーバント×サービス』,株式会社スクウェア・エニックス,第3巻 p.62,2013

\bibitem{Ahogation}
野地 遼一,阿部 隼多,伊藤 貴洋,諸戸 貴志,濱川 礼:触覚型デバイスによる感情表現システムAhogation,研究報告ゲーム情報学(GI),2016

\bibitem{Hairware}
\begin{flushleft}
Katia Vega, Marcio Cunha, Hugo Fuks: Hairware: The Conscious Use of Unconscious Auto-contact Behaviors, IUI '15: Proceedings of the 20th International Conference on Intelligent User Interfaces, pp.78\UTF{2013}86, 2015, https://doi.org/10.1145/2678025.2701404
\end{flushleft}

\bibitem{HairIO}
Christine Dierk, Sarah Sterman, Molly Jane Pearce Nicholas, Eric Paulos: H\UTF{00E4}irI\UTF{00D6}: Human Hair as Interactive Material, TEI '18: Proceedings of the Twelfth International Conference on Tangible, Embedded, and Embodied Interaction, pp.148\UTF{2013}157, 2018, https://doi.org/10.1145/3173225.3173232

\bibitem{LH}
Bing Li, Dawei Zheng, Yujia Lu, Fangtian Ying, Cheng Yao: LightingHair Slice: Situated Personal Wearable Fashion Interaction System, CHI EA '17: Proceedings of the 2017 CHI Conference Extended Abstracts on Human Factors in Computing Systems, pp.1824\UTF{2013}1828, 2017, https://doi.org/10.1145/3027063.3053093

\bibitem{Stail}
佐藤 大貴,三武 裕玄,長谷川 晶一:メッシュチューブとワイヤ駆動を用いたS字を描ける装着型猫のしっぽデバイス,エンタテインメントコンピューティングシンポジウム2015論文集,2015

\bibitem{Qoobo}
ユカイ工学:Qoobo,2018.参照ページ:“【公式ストア】Qoobo(クーボ)心を癒やす、しっぽクッション。”, ユカイ工学 オンラインストア,https://store.ux-xu.com/products/qoobo(2022年3月20日最終閲覧)

\bibitem{tentacle}
中安 翌:“触手アクチュエータの作り方”, lecture.nakayasu.com,2018年8月23日,http://lecture.nakayasu.com/?p=2896(2022年3月17日最終閲覧)

\bibitem{sg90}
秋月電子通商:“マイクロサーボ9g SG\UTF{FF0D}90”,秋月電子通商-電子部品・ネット通販,https://akizukidenshi.com/catalog/g/gM-08761/(2022年3月20日最終閲覧)

\bibitem{research_watch}
\begin{flushleft}
株式会社MM総研:“スマートウォッチの国内販売台数が200万台を突破 ≪ プレスリリース ”,株式会社MM総研,2021年9月16日,https://www.m2ri.jp/release/detail.html?id=508(2022年3月20日最終閲覧)
\end{flushleft}

\bibitem{research_mobileOS}
Statcounter Global Stats:“Mobile Operating System Market Share Japan”, Statcounter Global Stats, 2022年2月,https://gs.statcounter.com/os-market-share/mobile/japan(2022年3月20日最終閲覧)

\bibitem{research_mobileOS_2}
MMD研究所:“2021年12月スマートフォンOSシェア調査”,MMD研究所,2021年12月14日,https://mmdlabo.jp/investigation/detail\_2012.html(2022年3月20日最終閲覧)

\bibitem{A Dog Tail for Utility Robots}
Singh, A., Young, J.E, A Dog Tail for Utility Robots: Exploring Affective Properties of Tail Movement. In: Kotz\UTF{00E9}, P., Marsden, G., Lindgaard, G., Wesson, J., Winckler, M. (eds) Human-Computer Interaction \UTF{2013} INTERACT 2013. INTERACT 2013, Lecture Notes in Computer Science, vol 8118. Springer, Berlin, Heidelberg, 2013, https://doi.org/10.1007/978-3-642-40480-1\_27

\bibitem{Costail}
\begin{flushleft}
Cosgear:“Costail - Robotic Tail \UTF{2013}”, Cosgear, https://cosgear.co/products/buycostail?variant=31754620567651(2022年3月20日最終閲覧)
\end{flushleft}

\bibitem{noise}
全国環境研協議会 騒音調査音小委員会:“騒音の目安 (都心・近郊用)”,環境省,2018年8月14日,https://www.env.go.jp/air/souon\_meyasu\_1.pdf(2022年3月20日最終閲覧)


\end{thebibliography}

\end{document}
