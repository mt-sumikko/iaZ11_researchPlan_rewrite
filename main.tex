% 更新日2022年1月26日 by Tetsuaki Baba
\documentclass[a4paper]{jsarticle}
% 学部,修士は \documentclass[a4paper]{jsarticle}を利用して,\section で論文を構成してください.
% 博士は \documentclass[a4paper]{jsreport} を利用して, \chapter で論文を構成してください.

\usepackage{iapaper}
\usepackage[dvipdfmx]{graphicx}

\begin{document}
% 卒業論文の場合は \degreethesis
% 修士論文の場合は \masterthesis
% 博士論文の場合は \doctorthesis
\masterthesis %このサンプルでは卒業論文にしています


\title{研究計画(改訂版)}
\date{2022 年度}
\advisor{向井智彦} % 指導教員名を入れる
\IDnumber{22864614}
\Mauthor{隅山侑衣子} % 自分の名前を入れる.修士の場合は \Mauthor{氏名} に変更する.
\submissiondate{2022年7月1日}
\maketitle


\pagenumbering{roman} % 要旨はRoman書体で表示
\setcounter{page}{1} % 1から振り直す



\newpage

\pagenumbering{arabic}  % 論文本体はArabicで表示
\setcounter{page}{1} % ページ番号を1から振り直す

% 修士論文や博士論文等で記述ページが多い場合は,\section(節) ではなく,\chapter(章) を使ってください

\section{テーマ}
着用が目立たないウェアラブルデバイスによる髪の動的表現の開発と評価

\section{そのテーマを希望する理由・動機}
本研究テーマを希望する動機は、日常的な場面で利用可能な、感情や意図の表現手段を増やすことである。
人間の感情や意図は、発話や顔面の表情、身ぶり手ぶり等によって表現されるが、病気や習慣等の影響でこれらの手段をうまく使えない人もいる。そこで、犬や猫の尾のように感情や意図を表すことが一般に認知されている物を模倣すれば、表現の幅が広がると考える。ただし、模倣対象が人間の外観に調和しない場合は目立ち、使用シーンが限られる。そこで本研究では、目立たないことを条件とし、頭髪の毛束を尾に見立て、尾の表現を模倣する。


\section{そのテーマに関する研究の状況や背景}
表現拡張系デバイスの例に、neurowearチームによる necomimi \cite{necomimi}やshippo \cite{shippo}がある。前者は猫の耳を、後者は尻尾をモチーフとしており、整体情報をセンシングして動く。その他の例を含め、表現拡張系デバイスの多くは装着して目立つことを前提にデザインされてきた。それゆえ、目立たずに利用できない課題がある。感情等の表現対象が明確に現れることは,表現方法を拡張する目的にすらなり得る。しかし、表現媒体の様相として、その存在が目立つことに必然性はないと考える。

そこで本研究では、耳や尻尾といった外観ごと身につけるのではなく、表現媒体には人に備わっている頭髪を採用し、それを動かす機能を与えるデバイスを提案する。頭髪を採用する理由は、普段のコミュニケーションにおける視認性に優れているためである。顔付近に表現手段が増えることは、人間のコミュニケーションが主に顔を見て行われることや、マスクの着用、および肩から上のみが映るビデオ通話を多用する現況を踏まえても、ハンディキャップの有無を問わず有用であると考える。さらに、漫画やアニメーションにおいて,感情を表すという既知の表現があるため、髪と感情の間には既に一定のイメージが形成されていると言える。

頭髪をインタフェースとする関連研究では、その多くが髪に触れることによる情報の入力を扱っている\cite{smartWig}\cite{Hairware}\cite{HairIO}。出力による表現は発光\cite{LH} \cite{VariWig}や伸縮\cite{HairIO}に限られている。また、これらの研究で採用された形状の多くが、細い毛束か三つ編みである。つまり、髪による動的出力は未成熟の分野であり、技術を適用する髪形にも制限がある状況である。

\section{研究の目的} 
本研究の目的は、毛束による動的表現を目立たないデバイスで実現し、その有用性を評価することである。とくに、編み込まない髪型を対象とし、毛束を尾のように振るといった、これまで実践されてこなかった形状および動きのバリエーションを展開する。そして、動く髪をがどのような文脈、状況で受け入れられ、活用できるのかを考察する。

\section{考えられる研究の方法} 
髪を動かす技術的な検証段階では、ヘアエクステンションやかつらを用いた試作の上、有効な場合は地毛への展開を試みる。なお、目立たず、かつ身につけられるという条件を達成するには、小型・軽量・静音化が必須である。これらの対応は高い技術力を要する上、試作の初期段階で考慮するには不向きな場合がある。そのため、目的に応じて条件を緩和の上、表現の開発に重点を置くことも考えられる。つまり、目的を細分化して施策を繰り返すことで、実現可能な落とし所を探るプロセスを踏む必要がある。

また、評価の段階では、プロトタイプを日常生活で用いることや、体験型のワークショップを通じて実証実験を行う。ユーザや観察者側の意見・感想を収集して、操作方法や挙動を評価し、改善する。

\section{その研究の特色}
身体をハードウェア込みで拡張するのではなく、既存の身体部位に対して機能拡張を行うことをコンセプトとする点が特色である。このアプローチにより、デバイス着用における心身への負荷を軽減できる。また、表現拡張系デバイスの多くが目立つ外観をしているのに対して、目立たないデバイスを目指す点、また髪を動かすことが日常的な文脈でどのように活かし得るのかを探求する点が特徴である。


\begin{thebibliography}{999}

\bibitem{necomimi}
 neurowear: necomimi, 2012・2021. 参照ページ:“necomimi”, neurowear,https://neurowear.com/necomimi/(参照2022年3月20日)

\bibitem{shippo}
neurowear: shippo, 2012. 参照ページ:“脳波で動くしっぽ “shippo” ”,KILUCK:機楽株式会社,2012年10月15日,http://www.kiluck.co.jp/?p=61\&lang=ja(参照2022年3月20日)
  
\bibitem{smartWig}
Tobita, H., and Kuzi, T. SmartWig: wig-based wearable computing device for communication and entertainment. In Proceedings of the International Working Conference on Advanced Visual Interfaces (2012), AVI ’12, Association for Computing Machinery, pp. 299\UTF{2013}302. https://doi.org/10.1145/2254556.2254613

\bibitem{Hairware}
Katia Vega, Marcio Cunha, Hugo Fuks: Hairware: The Conscious Use of Unconscious Auto-contact Behaviors, IUI '15: Proceedings of the 20th International Conference on Intelligent User Interfaces, pp.78\UTF{2013}86, 2015, https://doi.org/10.1145/2678025.2701404

\bibitem{HairIO}
Christine Dierk, Sarah Sterman, Molly Jane Pearce Nicholas, Eric Paulos: H\UTF{00E4}irI\UTF{00D6}: Human Hair as Interactive Material, TEI '18: Proceedings of the Twelfth International Conference on Tangible, Embedded, and Embodied Interaction, pp.148\UTF{2013}157, 2018, https://doi.org/10.1145/3173225.3173232

\bibitem{LH}
Bing Li, Dawei Zheng, Yujia Lu, Fangtian Ying, Cheng Yao: LightingHair Slice: Situated Personal Wearable Fashion Interaction System, CHI EA '17: Proceedings of the 2017 CHI Conference Extended Abstracts on Human Factors in Computing Systems, pp.1824\UTF{2013}1828, 2017, https://doi.org/10.1145/3027063.3053093

\bibitem{VariWig}
Damien Brun,  Jonna H\UTF{00E4}kkil\UTF{00E4}. V\UTF{00E4}riWig: Interactive Coloring Wig Module. ISWC '21: 2021 International Symposium on Wearable Computers, September 2021,  pp. 166\UTF{2013}169. https://doi.org/10.1145/3460421.3478832

\end{thebibliography}

\end{document}
